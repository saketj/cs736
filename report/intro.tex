\section{Introduction}
\label{sec-intro}
Most modern systems have been built with the disk or SSDs as the persistent storage medium and the DRAM for temporary fast byte-addressable memory usage. The data that resides in volatile memory can be lost on crashes or power failure. Moreover, with the emergence of modern storage technologies and new byte addressable persistent memory technologies which offer much lower write and read latencies compared to flash memory at a slightly higher cost, it is imperative that future systems would move towards the usage of byte-addressable non volatile memory as the primary storage with the disk as the secondary storage medium. In this paper we have implemented and evaluated one such system. There has always been a trade-off between data durability and data read/write performance when deciding between a fast, volatile storage medium like DRAM versus a slow, persistent storage medium like disk. Persistent memory technologies like phase change spin-torque transfer RAM (STT-RAM), phase change memory (PCM), resistive RAM (ReRAM), and 3D XPoint memory technology can solve both of these problems[Suzuki].

One of the earliest file systems to be proposed for non-volatile memory systems was the byte-addressable persistent file system (BPFS) [BPFS] by a group of researchers working at Microsoft Research. The BPFS paper showed that simply running a disk-based traditional file system on top of persistent memory is not enough to provide high performance. BPFS proposes adding two new hardware primitives- 8-byte atomic writes and epoch barriers- to enforce atomicity and write ordering. It also proposed to modify the file system data structure to a new tree-based data structure  could support in-place updates. A new technique called short-circuit shadow paging as a copy-on-write mechanism. BPFS achieves fine-grained access to persistent data at a much improved performance, while at the same time providing strong reliability and safety guarantees.

The primary focus of this paper is to examine the benefit of such a multi-tiered storage hierarchy consisting of NVRAM and Disk or Flash on File Systems. The Paper "Better I/O Through Byte-Addressable, Persistent Memory" examined the implementation of a new file system for BPRAM called BPFS which performed faster than any other file system designed of block based storage devices. But the BPFS[] paper does not consider the fact that the relatively high cost/byte of NVM could hinder such a filesystem from being pervasively used since commodity systems will not have sufficient non volatile memory storage.

Due to the inability to store all data blocks in NVRAM, we have explored the usage of Anti-Caching [Paper] in BPFS.Anti-caching entails the storage of the more frequently accessed data blocks in the persistent byte addressable memory while storing the rest of the blocks on disk or flash memory. In short, while trying to read a disk block, the file system first checks to see if the block is already in the NVRAM and fetches the block from disk and evicts blocks from NVRAM to make room for the new block if needed using an LRU policy. Since double buffering of blocks is a waste of resourcs, the usage of NVRAM as a persistent store also requires us to pin certain blocks and maintain an eviction policy between itself and the secondary storage device.

The primary motivation for building such a system is to come up with a flexible design that scales itself well irrespective of the size of the storage tier, i.e capacity of NVRAM and capacity of disk or flash. This approach is very similar to the virtual memory swapping in operating systems wherein when the amount of data exceeds available memory, cold blocks are written to disk giving an impression to the file system that the amount of memory in NVM is infinite but at the same time provides the best possible performance with the amount of NVM available. This has been achieved through the use of anti-caching. 


