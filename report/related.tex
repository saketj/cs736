
\section{Related Work}

Current research on file systems for NVM is focused on maximizing file system performance by exploiting higher I/O throughput guarantees of the byte-addressable persistent NVM \cite{c10,c8,c3,c5}. While a majority of the research in this area has tried to solve the problem of consistency, write ordering and atomicity associated with implementing file systems for NVM, there has been some related works in accelerating the performance of existing file systems using NVM \cite{c11, c12}. While many file systems for NVM have been proposed as a replacement for traditional disk and flash-based file systems, our work proposes to enhance these file systems to function alongside the existing traditional disk and flash-based file systems. 
NOVA \cite{c8} takes a different approach than the byte-addressable persistent file system (BPFS) \cite{c10} as it implements a log structured file system that provides better concurrency by using separate logs for each inode and guarantees consistency by storing the journals in NVM. The overhead of garbage collection in LFS is also reduced in NOVA by exploiting the low random write latency of NVM when compared to the poor sequential write access latencies of SSDs and disks.

PMFS \cite{c3} is similar to other persistent memory file systems but provides transparent large page support for faster memory-mapped I/O and provides a way for applications to specify file sizes hints causing PMFS to use large pages for files data nodes. Aerie \cite{c5} provides a flexible file system architecture that aims to avoid the overhead of trapping into kernel while reading or writing to a file. It achieves this by splitting the functionality across different layers - a kernel layer that handles protection, allocation and addressing, a user library that directly accesses NVM for file reads, file writes and metadata reads, and a trusted service that coordinates updates to metadata.

Lv et. al. \cite{c1} propose a strategy named Hotness Aware Hit (HAT) for efficient buffer management in flash-based hybrid storage systems by dividing pages into three hotness categories: hot, warm and cold. In general, the hot, warm and cold pages are kept in main memory, flash and hard disk respectively according to the page access history and pages hotness. Our approach employs a similar strategy to determine the hotness of data and adapts it with the changes in the usage pattern by employing a multi-tier page reference history. However, our approach differs from this work because we do not use NVM merely for buffering of data; we use it as one of the storage devices in the multi-tier storage system with strategies to make placement decision during the runtime. 

Ghandeharizadeh et. al. \cite{c2} propose a way to use knowledge about the frequencies of read and write requests to individual data items in order to determine the optimal cache configuration given a fixed budget. This paper considers both tiering and replication of data across the selected choices of storage media. Our approach is different in following ways- we only consider tiering approach and do not replicate data across the storage layers, and the decision of placement of data is done online based on the information such as available space in NVM, size of the data and last access time. A separate background process moves cold data from higher storage level (NVM) to lower level (SSD / disk).
 
