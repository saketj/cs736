\begin{abstract}
Fast non-volatile memories (NVM) are byte addressable, offer latencies close to DRAM and have densities better than DRAM that allow applications to persist their data close to memory latencies, but at a higher predicted cost per byte. The I/O latency of a disk-based filesystem is higher than a pure memory-based filesystem, but memory-based filesystems are limited by their storage cost. We propose a hybrid approach, where we use NVM as a primary storage and use flash/disk-based for secondary storage. This paper provides the design and implementation of a file system built on top of an existing in-memory file system for a persistent storage tier that includes both NVRAM and Hard Disks, but at the same time provide latencies close to NVRAM at a lower cost.
\end{abstract}
