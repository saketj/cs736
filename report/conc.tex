\section{Conclusion}
\label{sec-conc}

Fast non-volatile memories (NVM) are byte addressable and offer latencies close to DRAM, but incur a higher predicted cost per byte. Combining the low latency of NVM and low cost of disk-based storage without regressing on the performance maximizes the best of the two worlds. In this paper, we present design, implementation and analysis of multi-tiered storage hierarchy consisting of NVRAM and Disk (or Flash) for File Systems, thereby providing a virtual impression of infinite NVM memory as well as the cheap upgrade in the disk performance with the amount of NVM available through the use of anti-caching mechanism.


\section{Future Work}
\label{sec-conc}

We identify several opportunities to extend this work in future. In this paper we did not get to do a thorough analysis on impacts and effects of hotness and coldness of data using multiple workloads and macro-benchmarks. We believe that doing so will reveal several potential areas to improve the system both functionally and architecturally. It is very important that we extensively evaluate and analyze crash consistency and reliability issues for this system in depth before extending the system. Lastly, since we simulated the NVM in DRAM, we did not get to evaluate the effects of persistent LRU cache on the performance of the system. Persistent LRU cache could have a significant effect on different aspects of the file system, such as boot time, that remains yet to be studied.
