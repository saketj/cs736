\section{Conclusion}
\label{sec-conc}

Future commodity systems can potentially include NVM as an integral part of the storage tier just as current systems include DRAM. However, the capacity of NVM is limited and may vary drastically in addition to being more expensive. AC-BPFS provides a file system design that scales well and tunes itself according to the capacity of the individual memory devices in the storage tier, while also providing similar performance and consistency guarantees without replicating data across storage devices. Further, AC-BPFS offers a modular and flexible system where the policies and mechanisms used in both anti-cache manager and disk manager can be modified independently, as per the workload and characteristics of the system.

\section{Future Work}
\label{sec-conc}

We identify several opportunities to extend this work in future. In this paper we did not get to do a thorough analysis on impacts and effects of hotness and coldness of data using multiple workloads and macro-benchmarks. We believe that doing so will reveal several potential areas to improve the system both functionally and architecturally. It is very important that we extensively evaluate and analyze crash consistency and reliability issues for this system in depth before extending the system. Lastly, since we simulated the NVM in DRAM, we did not get to evaluate the effects of persistent LRU cache on the performance of the system. Persistent LRU cache could have a significant effect on different aspects of the file system, such as boot time, that remains yet to be studied.
