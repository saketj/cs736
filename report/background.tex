\section{Background}
\label{sec-bg}
In this section, we will briefly discuss the relevant background on BPFS, a file system for byte address persistent memory. BPFS is one of the earliest file systems to be proposed for non-volatile memory systems and the authors showed that simply running a disk-based traditional file system on top of persistent memory is not enough to provide high performance. We will first introduce the BPFS file system layout and also discuss why the design is limited in terms of the cost of storage and volume of storage. We will also briefly describe the techniques proposed in BPFS to achieve consistency and how they achieve atomicity and write ordering. We then describe the changes made to the persistent data structures and file system code to support a disk backed secondary storage in a subsequent section.

\subsection{BPFS Layout}
BPFS stores all the file system metadata and data in the form of a tree structure consisting of fixed-size blocks of 4KB in the non-volatile memory. The BPFS file system consists of three kinds of files namely inode file, directory file and data file with each of these inturn arranged as a tree. The inode file is represented by a tree structure with the root of the inode file serving as the root of the file system. The location of this root node is stored at a known location in the non-volatile memory. Each of the leaves in the inode file contain an array of inode structures with each inode structure representing either a file or a directory. This inode structure contain the pointer to the root of the file or directory that the inode represents.
